\documentclass[a4paper,11pt]{article}

\usepackage{natbib}
\bibliographystyle{plainnat}

\begin{document}

\section{Introduction}

define Phylogenetics\\
major Methods of phlyogenetics:\\ 
 - Phenetic distance based methods (Neighbour-joining)\\
 - Character based methods (Bayesian, MP and ML)\\
what can Phylogenetics do for us?\\

accuracy of Phylogenitic trees?\\
 - consistency (convergence on correct tree)\\
 - efficiency (rate of convergence with more data added)\\
 - robustness (how sensitive to violations of robustness)\\

missing data effects\\

molecular improvements\\

models\\

mass extinctions\\

fossils in phlyogenetics - 50 billion species, 99\% extinct \\
 - fossils can provide a strong basis of evolutionary classification, when adequate data is available \\
 - fossils can alter exisiting relationships and infer new ones\\
 - Break-up long branches\\
 - Stabalise uncertain polarities\\
 - Sequence events\\
 - Calculating rates\\

fossil calibrations\\
- node dating\\
- tip dating \\
- total evidence analysis (node + tip dating)\\

Node dating = most common method >2000 references\\
Node and tip dating = improves the resolution and greater support for ages/posteriors, can be less sensitive to priors\\

Yule process = assumes a constant rate of speciation across lineages (no extinction).\\

Birth-death process = at any point in time a lineage can speciate at rate λ (speciation rate) or go extinct with a rate of μ (extinction rate), connected by diversification rate and turnover.\\

Fossilized birth-death process = probability of the tree and fossil observations under a birth-death model. Improves inference of nodes. Better capture statistical uncertainty. Turnover and diversification rates now linked by fossil recovery rate and the probability of fossil recovery.\\

\section{Methods}

take two morphological data sets \\

Beck and Lee character matrix \citep{beck2014ancient} (102 taxa (47 unique), 421 characters (13 unique) combined with..\\  

Halliday and Goswami matrix \citep{halliday2016eutherian} (177 taxa (115 unique), 680 characters (222 unique).\\ 

(59 common species)\\

combine two morphological data sets using R \\
 - Assumptions: \\ 
 - Halliday character states/descriptiions updated Becks\\ 

 - Alterations:\\
 - Some characters from Beck and Lee were broken down in Halliday and Goswami’s matrix. \\
 - Some character states were also reversed between matrices. These were altered in the nexus and are marked in the character table. \\
 
Characters as: 

Author = Original author of character

OCN = Original character number (where found)

Name = First three letters of author, date of publication (last two digits of four digit year), code of character type, number identifier within character type. \\ 
 
Combined using R - Adding Becks unique taxa and characters to Hallidays matrix. \\ 

use composite matrix to run Bayesian analysis\\
create constraints using PaleoTree package in R\\
packages used: Devtools, Claddis, PaleoTree \\

parameters\\
 - outgroup Nanolestes =\\
 - delete taxa Laindon, Montamalestes, Kharmerungulatum, Alymlestes = Because these are unstable taxa\\
 - constrained the ingroup and root to seperate groups (1 and 2)\\
 - clockvarpr=igr (independent gamma rates) =it is the rate at which the variance of the effective branch length increases over time,\\
 - brlenspr=clock:fossilization (birth-death-rate)\\
 - fixed tip dating = Beck species tips dated as Becks dates, Hallidays additional species dated using fossilworks, for these two species (!!!!) ages were taken from (!!!!), Eomanis is from messel pit, messel pit age =  \\
 - constrain the combined tree by Becks original tree, then Hallidays original tree, then by both combined\\

 - Halliday's topology constraint\\
For the Halliday's constraint, we build a majority consensus tree from the topologies of [Halliday and Goswami 2015]\\ (http://onlinelibrary.wiley.com/doi/10.1111/bij.12731/abstract) without branch length (using `ape::consensus` in R).

 - Beck's topology constraint
For the Beck's constraint, we build a majority consensus tree from the 7 constrained topologies of [Beck and Lee 2014]\\ (http://rspb.royalsocietypublishing.org/content/281/1793/20141278) without branch length (using `ape::consensus` in R).
We selected the constrained trees only to avoid to much noise from the unconstrained topologies.

 - SuperTree's topology constraint
For the super tree topology, we build a simple supertree from the two topologies described above using [clann](http://chriscreevey.github.io/clann/).
We ran the supertree search on both trees using the clann default parameters (Most Similar Supertree searched with a Sub-tree Pruning Regrafting algorithm for 10 heuristic searches).

 - mcmc temp=0.15, 4 runs, 4 chanins, burnin of !!!!, number of generations = !!!! 

\section{Results} 

position/disparity of "added taxa"

\section{Discussion}

affect of adding data


 










 

\section{References}

\bibliographystyle{plainnat}
\bibliography{References/Biblo}

\end{document}