\documentclass[a4paper,11pt]{article}

\usepackage{natbib}
\bibliographystyle{plainnat}

\begin{document}

\section{title}

How Does Adding Data Change Node Age Estimation In Morphological Phylogenies

\section{Abstract}

\section{Introduction}

\subsection{Paragraph 1}
- (Broad) = what is the paper about and why is this interesting, why isn't there an answer already?\\
- Explanation of difficulties of phylogenetics as a broad topic.\\
- Accuracy of phylogenetics = consistency (convergence on correct tree), efficiency (rate of convergence with more data added), robustness (how sensitive to violations of robustness)\\
- Use of fossils for dating trees.\\
- Importance of node dating and divergence times.\\

\subsection{Paragraph 2}
- (Focusing)
- Mammals in phylogenetics.\\
- Mammal divergence estimates around K-pg.\\
- 

\subsection{Paragraph 3}
- (Focused)
- Hypotheses
- How is this going to be tested? 
- Why are we testing it in this way?

\subsection{Other Ideas}
- define Phylogenetics\\
major Methods of phlyogenetics:\\ 
- Phenetic distance based methods (Neighbour-joining)\\
- Character based methods (Bayesian, MP and ML)\\
what can Phylogenetics do for us?\\

accuracy of Phylogenitic trees?\\
- consistency (convergence on correct tree)\\
- efficiency (rate of convergence with more data added)\\
- robustness (how sensitive to violations of robustness)\\

missing data effects\\

molecular improvements\\

models\\

fossils in phlyogenetics - 50 billion species, 99\% extinct \\
- fossils can provide a strong basis of evolutionary classification, when adequate data is available \\
- fossils can alter exisiting relationships and infer new ones\\
- Break-up long branches\\
- Stabalise uncertain polarities\\
- Sequence events\\
- Calculating rates\\

fossil calibrations\\
- node dating\\
- tip dating \\
- total evidence analysis (node + tip dating)\\

Node dating = most common method >2000 references\\
Node and tip dating = improves the resolution and greater support for ages/posteriors, can be less sensitive to priors\\

Yule process = assumes a constant rate of speciation across lineages (no extinction).\\

Birth-death process = at any point in time a lineage can speciate at rate λ (speciation rate) or go extinct with a rate of μ (extinction rate), connected by diversification rate and turnover.\\

Fossilized birth-death process = probability of the tree and fossil observations under a birth-death model. Improves inference of nodes. Better capture statistical uncertainty. Turnover and diversification rates now linked by fossil recovery rate and the probability of fossil recovery.\\

\section{Methods}

\subsection{Data Combination}

1). Gather data in the form of two morphological data sets \\
 - Beck and Lee character matrix \citep{beck2014ancient} (102 taxa (47 unique), 421 characters (13 unique) combined with..\\  
 - Halliday and Goswami matrix \citep{halliday2016eutherian} (177 taxa (115 unique), 680 characters (222 unique).\\ 
 - (59 common species)\\

2). Create a super matrix in excel to standardise the data format and naming system.\\
 - Characters formatted as: \\
 - Author = Original author of character.\\
 - OCN = Original character number (where found).\\
 - Name = First three letters of author, date of publication (last two digits of four digit year), code of character type, number identifier within.\\ character type. \\ 
 - During creation of the super matrix, some descriptions of shared characters were different. In these cases Hallidays descriptions were taken.\\
 - Some of Becks original characters were broken down into subs, more specific characters, by Halliday.\\

3). R was used to combine these two data matrices (full code in supplementary?).\\
 - The character data from Beck and Halliday was inported to R using a modified ``ReadMorphNexus'' from the Claddis package.\\
 - The matrices were extracted and combined based on unique and common taxa.\\
 - A loop was written to place the added characters into the correct position in the super matrix.\\
 - Some shared characters had flipped states in Hallidays data, a loop was used and a function written (``flip.characters.R'') to re-order the states of these characters according to a list.\\

 4). A modified ``write.nexus.data'' function was used to export the finalised matrix.\\

 5). Paleotree package was used to create the two constraints used. After 
 - Halliday's topology constraint\\
For the Halliday's constraint, we build a majority consensus tree from the topologies of [Halliday and Goswami 2015]\\ (http://onlinelibrary.wiley.com/doi/10.1111/bij.12731/abstract) without branch length (using `ape::consensus` in R).
 - Beck's topology constraint\\
For the Beck's constraint, we build a majority consensus tree from the 7 constrained topologies of [Beck and Lee 2014]\\ (http://rspb.royalsocietypublishing.org/content/281/1793/20141278) without branch length (using `ape::consensus` in R).
We selected the constrained trees only to avoid to much noise from the unconstrained topologies.

\subsection{Tree Creation}

1). Trees were created using Mr Bayes (Nexus files available in Supp.)\\

2). parameters\\
- outgroup Nanolestes = why?
- delete taxa Laindon, Montamalestes, Kharmerungulatum, Alymlestes = Because these are unstable taxa\\
- constrained the ingroup and root to seperate groups (1 and 2)\\
- clockvarpr=igr (independent gamma rates) =it is the rate at which the variance of the effective branch length increases over time,\\
- brlenspr=clock:fossilization (birth-death-rate) = why?\\
- fixed tip dating = Beck species tips dated as Becks dates, Hallidays additional species dated using fossilworks, for these two species (!!!!) ages were taken from the age of the messel pit formation. = why?\\
- constrain the combined tree by Becks original tree, then Hallidays original tree, [then by both combined (!!!!)]\\
- 2 runs = why?\\
- 4 chains = why?\\
- burnin of (!!!!) = why?\\

\subsection{Other}

combine two morphological data sets using R \\
- Assumptions: \\ 
- Halliday character states/descriptiions updated Becks\\ 

- Alterations:\\
- Some characters from Beck and Lee were broken down in Halliday and Goswami’s matrix. \\
- Some character states were also reversed between matrices. These were altered in the nexus and are marked in the character table. \\
 
Combined using R - Adding Becks unique taxa and characters to Hallidays matrix. \\ 

use composite matrix to run Bayesian analysis\\
create constraints using PaleoTree package in R\\
packages used: Devtools, Claddis, PaleoTree \\

- What is the overall design of study?\\
- What are the variables and how do they relate to the hypotheses?\\
- How/where get the data?\\
- What are the characteristics of the data set / experiment -- how many observations, how many replicates etc.\\
- General procedures, if any, that are true in all of the analyses (e.g., transformation of data, model checking, how models were compared)\\
- How did you test the hypotheses, in the logical order outlined in the introduction (i.e., from the general to the specific)? Make sure you show that your tests are appropriate.\\

\section{Results} 

\subsection{Plot of node ages against data sets}

\subsection{Table of raw node ages}

\section{Discussion}

- What’s the main thing we know now that we didn’t know before? What affect did adding data have on the node ages?\\

- What’s the chain of logic and results that means we know this answer?\\

- How does this affect our future research? What are the possible implications of this answer?\\

- Caveats\\
- Time constraints limited the number of generations trees could run for.\\
- Limited time also meant heavy constraints were needed and fewer parameter combinations could be tested.\\
- What might be done about them? (Very important in a project write-up -- What would you do differently if you were doing the project again or had more time?)\\

- What future work could build more broadly on what we’ve found?\\
- How may this study in this area be of interest to other work in other areas, or other systems.\\

- Link\\

\section{Supplementary}

- R code\\
- Trees\\{}
- Mb command files\\ 

\section{Possible Papers}

\citep{halliday2016eutherian}
\citep{halliday2016impact}
\citep{halliday2017resolving}
\citep{halliday2016eutherians}
\citep{beck2014ancient}
\citep{arcila2015evaluation}
\citep{near2004assessing}
\citep{lee2015morphological}
\citep{puttick2016dating}
\citep{pyron2016novel}
\citep{turner2017empirical}
\citep{heath2014fossilized}
\citep{wiens2004role}
\citep{nabhan2011impact}

\bibliographystyle{plainnat}
\bibliography{References/Biblo}

\end{document}